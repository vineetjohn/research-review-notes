\documentclass[12pt]{scrartcl}

\usepackage[utf8]{inputenc}
\usepackage[T1]{fontenc}
\usepackage{amsmath}
\usepackage{hyperref}


\hypersetup{
    colorlinks=true,
    citecolor=blue
}

\begin{document}

\title{Adversarial Generation of Natural Language}
\author{}
\date{}
\maketitle

\section{Main Idea}
The paper tries to leverage the success of GANs in the computer vision domain to generate language. The authors attempt to address the discrete space problem without the usage of gradient estimators. They also intend to generate language based on context-free grammars and also from conditional generation using sentence attributes. \cite{subramanian2017adversarial}

\section{Method}
  \begin{itemize}
    \item 
  \end{itemize}

\section{Observations}
  \begin{itemize}
    \item 
  \end{itemize}

\bibliographystyle{unsrt}
\bibliography{adversarial-generation-of-natural-language}

\end{document}
