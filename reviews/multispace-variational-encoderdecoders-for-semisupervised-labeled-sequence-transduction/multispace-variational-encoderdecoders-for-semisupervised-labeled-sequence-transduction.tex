\documentclass[12pt]{scrartcl}

\usepackage[utf8]{inputenc}
\usepackage[T1]{fontenc}
\usepackage{amsmath}
\usepackage{hyperref}


\hypersetup{
    colorlinks=true,
    citecolor=blue
}

\begin{document}

\title{Multi-space Variational Encoder-Decoders for Semi-supervised Labeled Sequence Transduction}
\author{}
\date{}
\maketitle

\section{Main Idea}
  The general idea \cite{zhou2017multi} seems similar to style-transfer in text. Labeled sequence transduction is just a roundabout way of saying that a source text $x^{(s)}$ is to be transformed into a target text $x^{(t)}$ such that $x^{(t)}$ is conditioned on the labels $y^{(t)}$.

\section{Background}
\begin{itemize}
  \item The morphological reinflection problem tries to change a sequence of characters of an inflected word. For example, convert "playing" into "played", given a set of labels $y^{(t)}$ such that $y^{(t)}_{pos}=verb$ and $y^{(t)}_{tense}=past$
\end{itemize}

\section{Method}
  \begin{itemize}
    \item 
  \end{itemize}

\section{Observations}
  \begin{itemize}
    \item 
  \end{itemize}

\bibliographystyle{unsrt}
\bibliography{multispace-variational-encoderdecoders-for-semisupervised-labeled-sequence-transduction}

\end{document}
