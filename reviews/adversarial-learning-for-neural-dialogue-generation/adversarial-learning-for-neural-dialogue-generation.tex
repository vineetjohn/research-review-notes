\documentclass[12pt]{scrartcl}

\usepackage[utf8]{inputenc}
\usepackage[T1]{fontenc}
\usepackage{amsmath}
\usepackage{hyperref}


\hypersetup{
    colorlinks=true,
    citecolor=blue
}

\begin{document}

\title{Adversarial Learning for Neural Dialogue Generation}
\author{}
\date{}
\maketitle

\section{Main Idea}
  The authors formulate this dialogue model as a reinforcement learning problem. The network used is a Generative Adversarial Network. The discrimnator object is the same as a Turing test predictor i.e. classifies whether the dialogue response is human or machine-generated. The goal is to improve to improve the generator to the point where the discrimnator has trouble distinguishing between human and machine-generated responses. \cite{li2017adversarial}.

\section{Method}
  \begin{itemize}
    \item 
  \end{itemize}

\section{Observations}
  \begin{itemize}
    \item 
  \end{itemize}

\bibliographystyle{unsrt}
\bibliography{adversarial-learning-for-neural-dialogue-generation}

\end{document}
