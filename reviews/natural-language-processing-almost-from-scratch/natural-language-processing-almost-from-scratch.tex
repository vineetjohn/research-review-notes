\documentclass[12pt]{scrartcl}

\usepackage[utf8]{inputenc}
\usepackage[T1]{fontenc}
\usepackage{amsmath}
\usepackage{hyperref}


\hypersetup{
    colorlinks=true,
    citecolor=blue
}

\begin{document}

\title{Natural Language Processing (almost) from Scratch}
\author{}
\date{}
\maketitle

\section{Main Idea}
  The paper \cite{collobert2011natural} attempts to train a generic single learning system for multi-task learning. The taks include Part-of-Speech (POS) tagging, chunking, Named Entity Recognition (NER) and Semantic Role Labeling (SRL). The authors intend to acheive this without hand-engineering task-specific features, and instead rely on a large amount on unlabeled data.

\section{Method}
  \begin{itemize}
    \item 
  \end{itemize}

\section{Observations}
  \begin{itemize}
    \item 
  \end{itemize}

\bibliographystyle{unsrt}
\bibliography{natural-language-processing-almost-from-scratch}

\end{document}
