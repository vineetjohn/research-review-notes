\documentclass[12pt]{scrartcl}

\usepackage[utf8]{inputenc}
\usepackage[T1]{fontenc}
\usepackage{amsmath}
\usepackage{hyperref}


\hypersetup{
    colorlinks=true,
    citecolor=blue
}

\begin{document}

\title{Natural Language Processing (almost) from Scratch}
\author{}
\date{}
\maketitle

\section{Main Idea}
  The paper \cite{collobert2011natural} attempts to train a generic single learning system for multi-task learning. The taks include Part-of-Speech (POS) tagging, chunking (CHUNK), Named Entity Recognition (NER) and Semantic Role Labeling (SRL). The authors intend to acheive this without hand-engineering task-specific features, and instead rely on a large amount on unlabeled data. They also wish to avoid baselines that have been created using differently labeled data.
  
\section{Background}
  \begin{itemize}
    \item The state-of-the-art (SoTA) system for POS tagging uses bidirectional sequence decoders (Viterbi algorithm) and maximum entropy classifiers to determine, which among a set of pre-defined tags, can be attributed to a token.
    \item Chunking is essentially the same as POS-tagging, but for phrases instead of single words. SoTA for chunking uses pairwise SVM-classifiers, for which the features were word-contexts. Matrix SVD based methods have also been successfull.
    \item For NER, the SoTA is a linear model combined with Viterbi decoding, where the features include the tokens themselves, the POS tags, CHUNK tags, suffixes and prefixes.
    \item SRL is similar to obtaining an entity-relation model from unstructured data (text). SoTA on SRL are pasrse trees, CHUNK and POS tags, voice, types of verb etc. in combination with context-window classifiers.
  \end{itemize}

\section{Method}
  \begin{itemize}
    \item 
  \end{itemize}

\section{Observations}
  \begin{itemize}
    \item One pertinent question is whether multi-task learning is only effective if the subtasks are not completely orthogonal to each other.
  \end{itemize}

\bibliographystyle{unsrt}
\bibliography{natural-language-processing-almost-from-scratch}

\end{document}
